\documentclass[11pt]{article}

\usepackage[utf8]{inputenc}
\usepackage[T1]{fontenc}

\usepackage[a4paper, left=2cm, right=2cm, top=3.5cm, bottom=3.5cm]{geometry}
\usepackage[french]{babel}

% Paragraph spacing
\setlength{\parskip}{1em}

% Fancy headers
\usepackage{fancyhdr}

% Captions for subfigures
\usepackage{subcaption}


% Code highlighting
\usepackage{minted}

% Footnote inside a caption
\usepackage{fnpos}
\usepackage{ftnxtra}

% Maths
\usepackage{amsmath}
\usepackage{amssymb}

% Todo notes
\usepackage{todonotes}

% Table of contents for bibliography
\usepackage[nottoc]{tocbibind}

% Inline monospace font
\def\code#1{\texttt{#1}}

% Figures
\usepackage{graphicx}

% Draw figures
\usepackage{tikz}

% Code listing
\usepackage{listings}

% Tikz node rotation
\usetikzlibrary{positioning}

% Turing machine
\usetikzlibrary{chains,fit,shapes}

% Usage: \rotnode[options]{rotation}{text}
\newcommand\rotnode[3][]{%
\node [#1, opacity=0.0] (tmp) {#3};
\node [draw, rotate around={#2:(tmp.center)}] at (tmp) {#3};
}

% Clickable links
\usepackage{hyperref}
% Table of contents depth
\setcounter{tocdepth}{2}

% Inline code
\usepackage{listings}
\usepackage{color}

\title{Systèmes d'exploitation}

\author{William SCHMITT \& Othmane AJDOR}
\date{2018-2019}

\begin{document}
\maketitle

\section{TD1 - Dekker}
Avec deux processeurs:
\begin{lstlisting}[frame=single]
	debut_SCC();
		load $i, %r0
		inc %r0
		store %r0, $i
	fin_SCC();
\end{lstlisting}
Après l'execution des instructions, i=1|2.\\
Ce qu'on veut avoir c'est obtenir i=2 et pas i=1

\subsection{Zero}
En utilisant cette fonction:
\begin{lstlisting}[frame=single]
	bool occ=false; //occupe
	debut_SCC(){
		alpha: 
			if (occ) 
				goto alpha; 
			occ = true;
	}

	finSCC(){
		occ = false;;
	}
\end{lstlisting}
On n'obtient pas d'exclusion mutuelle.

\pagebreak

\subsection{One}
\begin{lstlisting}[frame=single]
	bool occ = false;
	\\ecrire avant le test
	
	debut_SCC(){
		int old = occ; // local var
		occ = true;
		alpha : if(old)
					old = occ;
				goto alpha;
	}

	finSCC(){
		occ = false;
	}
\end{lstlisting}
Pas d'exclusion mutuelle.

\subsection{Two}
\begin{lstlisting}[frame=single]
	int tour = 0;
	p; // num de processeur {0;1}
	q = 1 - p; // num de l'autre processeur

	debut_SCC(){
		alpha:
			if (tour != p){
				goto alpha;
			}
	}

	finSCC(){
		tour = 1 - p;
	}
\end{lstlisting}
Exclusion mutuelle OK, par contre, le bout de code cause un problème de famine.\\
L'autre processeur ne pourra jamais executer la section critique.

\pagebreak

\subsection{Three}
\begin{lstlisting}[frame=single]
	bool access[2] = {false};
	//Accede a acces[] avec p/q <-annoncer ma demande avant la boucle 

	debut_SCC(){
		acces[p] = true;
		alpha:
			if (acces[q])
				goto alpha;
	}

	finSCC(){
		acces[p] = false;
	}
\end{lstlisting}
Exclusion mutuelle OK.\\
Si les deux processeurs executent le code en même temps, ils rentrent dans une boucle infinie.\\
Probleme de \textbf{dead\_lock}, les deux processeurs ne font plus rien.

\subsection{Four}
\begin{lstlisting}[frame=single]
	bool access[2] = {false, false};
	// Three et renoncer a sa demande
	// On recommence tout

	debut_SCC(){
		alpha:
			acces[p] = true;
			if (acces[q])
				acces[p] = false;
				// en reseau recursive doubling
				sleep(random); 
				goto alpha;
	}

	finSCC(){
		acces[p] = false;
	}
\end{lstlisting}
Probleme de \textbf{dead\_lock} si synchrone
Il faut qu'ils arrivent en même temps et qu'ils restent synchrones pour que ça marche.

\pagebreak

\subsection{Five}

\begin{lstlisting}[frame=single]
	bool access[2] = {false, false};
	int tour;
	// Comme la Four
	// Celui dont ce n'est pas le tour renonce

	debut_SCC(){
		alpha:
			acces[p] = true;
			if (p != tour && acces [q])
				acces[p] = false;
				sleep(random); 
				goto alpha;
	}

	finSCC(){
		acces[p] = false;
		tour = q;
	}
\end{lstlisting}
Pas d'exclusion mutuelle

\pagebreak

\subsection{Six}

\begin{lstlisting}[frame=single]
	bool access[2] = {false, false};
	int tour;
	// Comme la Four
	// Celui dont ce n'est pas le tour renonce

	debut_SCC(){
		acces[p] = true;
		
		alpha:
			if(acces[q]){
				if(tour==p){
					beta:
						if(access[q])
							goto beta;
				}
				else {
					acces[p] = false
					sigma: if(tour==q):
						goto sigma;
					goto alpha;
				}
			}

		}

	finSCC(){
		acces[p] = false;
		tour = q;
	}
\end{lstlisting}
Avant de commencer, je verifie que je suis dans le bon etat.\\
Meme si c'est mon tour, je vais attendre que l'autre renonce à acces.\\
L'autre cas c'est si je suis en conflit, je renonce à mon tour.\\
Une fois qu'il a fini, le tour devriendra le miens

\pagebreak

\subsection{Seven - La solution correcte de Peterson}
La solution correcte:
\begin{lstlisting}[frame=single]
	bool access[2] = {false, false};
	int dernier;

	debut_SCC(){
		acces[p] = true;
		dernier = p;
		alpha:
			if (acces[q]){
				if(dernier ==p)
					goto alpha;
			}

	finSCC(){
		acces[p] = false;
	}

\end{lstlisting}
\end{document}