\documentclass[11pt]{article}

\usepackage[utf8]{inputenc}
\usepackage[T1]{fontenc}

\usepackage[a4paper, left=2cm, right=2cm, top=3.5cm, bottom=3.5cm]{geometry}
\usepackage[french]{babel}

% Paragraph spacing
\setlength{\parskip}{1em}

% Fancy headers
\usepackage{fancyhdr}

% Captions for subfigures
\usepackage{subcaption}


% Code highlighting
\usepackage{minted}

% Footnote inside a caption
\usepackage{fnpos}
\usepackage{ftnxtra}

% Colored text, provides \textcolor{color}{text}
\usepackage{xcolor}

% Maths
\usepackage{amsmath}
\usepackage{amssymb}

% Todo notes
\usepackage{todonotes}

% Table of contents for bibliography
\usepackage[nottoc]{tocbibind}

% Inline monospace font
\def\code#1{\texttt{#1}}

% Figures
\usepackage{graphicx}

% Draw figures
\usepackage{tikz}

% Code listing
\usepackage{listings}

% Tikz node rotation
\usetikzlibrary{positioning}

% Turing machine
\usetikzlibrary{chains,fit,shapes}

% Usage: \rotnode[options]{rotation}{text}
\newcommand\rotnode[3][]{%
\node [#1, opacity=0.0] (tmp) {#3};
\node [draw, rotate around={#2:(tmp.center)}] at (tmp) {#3};
}

% remove extra space
\newcommand{\squeezeup}{\vspace{-4.5cm}}

% Clickable links
\usepackage{hyperref}
% Table of contents depth
\setcounter{tocdepth}{2}

% Inline code
\usepackage{listings}
\usepackage{color}

\title{Systèmes d'exploitation - Synchronisations III}

\author{Othmane AJDOR}
\date{2018-2019}

\begin{document}
\maketitle

\pagebreak
\tableofcontents
\pagebreak

\section{Grain}

"Grain" calcul/cout de gestion ou de communications

\section{Synchronisation dans les OS}
\subsection{Antiquité}

\subsubsection{Projet système}
On disposait d'un seul processeur, soit une activité à la fois.\\
Il ne faut interrompre des activités qui touchent à des ressources pour lancer une deuxieme. Pour cela, on met en place des sections critiques (sémaphores) dans l'OS et masquer les interruptions (!= exception -déclenché par le processeur lui meme-) pendant l'appel de l'OS.\\
Les interruptions ne seront prises en charges qu'à la sortie du bout de code comme on les a masquées.Les exceptions se font une à la fois, on peut pas faire deux codes (user et section critique dans l'OS).

\subsubsection{Linux 1.0}
Gestion de multi-processeurs.\\
Il faut que le noyau fonctionne meme si les processeurs veulent executer des fonctions du systeme.\\
Pour ce faire, on a choisi de placer un BKL (bit kernel lock) à chaque entrée de l'OS. Ce lock est une attente active avec les fonctions atomiques.\\
Ce modèle ne passe pas à l'echelle à cause de l'attente active.

\subsubsection{Aller plus vite}
Pour aller plus vite, on parrallelise le noyau => remplacer le BKL (unique) par des (beaucoup) petits locks.\\
Pour sortir les tests de synchronisation du noyau => test en mode user ou blocage, liberation de threads en mode kernel.

Attente active:
\begin{minted}[frame=single]{c}
int lock = false;
while (atomic_load(lock)); // /!\ BUG
    atomic_store(lock, 1); // /!\ BUG
SC
    atomic_store(lock, 0);
\end{minted}

\pagebreak

test\_and\_set() => tester une valeur, si elle vaut 0, je la met à 1 et je retourne l'ancienne valeur
\begin{minted}[frame=single]{c}
while (test_and_set(lock));
    SC
\end{minted}

CMP\_XCHG(v, old, new):
\begin{minted}[frame=single]{c}
if v==old
    mettre v a new
renvoyer l ancienne valeur de v

/* Verifie si la valeur de old a été modifiée par nous meme ou dans une autre 
partie du code */
atomic_inc(v){
    old = atomic_load(v);
    while(cmp_exchange(v, old, old+1) != old){
        old = atomic_load(v);
    }
}
\end{minted}

Load Linked:
\begin{minted}[frame=single]{c}
load_linked(v);
//...
//...    
bool store_conditionnal(v, new); // bool reussi ou pas

\end{minted}

\pagebreak

\subsubsection{Futex}

Quand on fait un wait, on attend à ce que v ait la valeur attendue
\begin{minted}[frame=single]{c}
futex_wait(&v, attendue); // attendu est la valeur liée au test atomic coté user
futex_wake(&v, nbWake); // Identique à signal où nbWake 
// est nombre de threads à reveiller
\end{minted}

\begin{minted}[frame=single]{c}
int mutex=0;
mtx_lock(*m){
    while((c_atomic_fetch_add(&m, 1)) != 0){ // coute 10 cycles
        futex_wait(&m, c); // coute 20k cycles
    }
}

unlock(*m){
    atomic_store(&m, 0); // Ajoute barriere memoire qui valide les caches (10 cycles)
    futex_wake(&m, 1); // reveiller un thread (20k cycles)
}
\end{minted}









\end{document}